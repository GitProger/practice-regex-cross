\documentclass[12pt]{article}
\usepackage[utf8]{inputenc}
\usepackage[main=russian,english]{babel}
\usepackage[left=20mm, top=15mm, right=20mm, bottom=10mm, nohead, nofoot]{geometry}

\usepackage{graphicx}
\usepackage{caption}
\usepackage{wrapfig}


% \def \var {value}
\newcommand{\ReportTheme}{Практика}
\newcommand{\ReportAuthor}{Данилевич Леонид, Лельчук Александр, 2022А класс}

\title{\bf \ReportTheme}
\author{\it \ReportAuthor}
\date{\today}

% 1. Титульный лист. (Одинаковая по форме. Взять в нетскуле)
% 2. Аннотация. (Около абзаца. Кратко, какая поставлена задача, какой результат получен. Делается в последний момент)
% 3. Оглавление.
% 4. Введение. (Где родилась задача. Ваша задача - подспорье в решении какой-то более глобальной. Зачем это всё нужно = Обоснование работы)
% 5. Постановка задачи. Сформулировать задачу так, чтобы можно было её решить математически/померить экспериментально/и т д.
% 6. Методика разрешения задачи. Как решается задача: математически/экспериментально/программно и т д. Подробности. С помощью каких механизмов.
% 7. Результаты. Что вы померили/насчитали/
% 8. Анализ результатов. Что означают результаты. Совпадают ли с предсказаниями. Что следует из результатов.
% 9. Список литературы. На что вы опирались.
% 10. Благодарности. С кем вы работали, кто вас консультировал, кто помогал и т п.

%\usepackage{mathptmx}

\begin{document}
   % \maketitle
   % титульный лист

    \begin{center}
    \large { {\bf Курсовая работа (отчет по практике) } } 
    
    Создание программы генериующей кроссворды из регулярных выражений \\
    
    \end{center}        
    \begin{flushright}
        Работу выполнили: \\
        Данилевич Леонид (2022А) \\
        Лельчук Александр (2022А) \\
        Научный руководитель: \\
        Дворкин Михаил Эдуардович
        Место прохождения практики: \\
        Лицей <<ФТШ>>
    \end{flushright}
    \begin{center}
        Санкт-Петербург, 2021
    \end{center}        
%----------------------------------------------------------------------------------------------------------
    \newpage % Аннотация
    
%----------------------------------------------------------------------------------------------------------
    \newpage % Оглавление

%----------------------------------------------------------------------------------------------------------
    \newpage % Введение
    \part*{Введение}
Регулярные выражения — формальный язык поиска подстрок в тексте и манипуляций с ними. Например, регулярному выражению «.*amp(le)?» соответствуют строки «Sample», «example», «lamp» и некоторые другие. С помощью регулярных выражений можно достаточно просто искать в тексте подстроки определённого формата и заменять их на соответсвующие им другие подстроки. Новичкам, изучающим регулярные выражения, для закрепления нового материала полезно решить такой (https://gregable.com/p/regexp-puzzle.html) кроссворд (изображение ниже). Конкретно этот экземпляр создан, как задание конкурса «MIT Mystery Hunt» 2013 года. Мы решили написать приложение, автоматически генерирующее подобные кроссворды, а также позволяющее их решать. Мы считаем, что такое приложение будет полезно многим людям, изучающим регулярные выражения, а для знакомых с ними оно будет просто интересно.

\textbf{ Методика решения задачи: }
Так как задача состоит в написании программы, то решается программно. Задача дробится на части:
1) Собственно, решение уже имеющегося кроссворда. Понадобится для оценки сложности кроссворда и и удостоверения единственности решения.
2) Генерация буквенного заполнения кроссворда. Должны образовываться строчки, подходящие под различные регулярные выражения. Максимизируется красота заполнения — субъективное свойство, включающее в себя регулярность кроссворда, т. е. различные повторения подстрок и английские слова.
3) Генерация регулярных выражений, лучшим образом задающих получившиеся строчки.
4) Оценка сложности, проверка единственности решения, путём программного решения (1) кроссворда

%----------------------------------------------------------------------------------------------------------
    \newpage % Постановка задачи

        
\end{document}

